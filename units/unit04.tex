\providecommand{\main}{..}
\documentclass[\main/notes.tex]{subfiles}

\begin{document}
	\addtocontents{toc}{\protect\newpage}
	\setcounter{chapter}{3}
	\chapter{Complex Numbers}
		\section{Complex Numbers}
			\begin{definition}{Complex Number}
				The number $i$ is defined such that:
				\begin{align*}
					i &= \sqrt{-1}\\
					i^{2} &= -1
				\end{align*}

				Then, if $a$ and $b$ are real numbers, a \concept{complex number} can be written in the form:
				\begin{align*}
					z = a + bi
				\end{align*}

				The number $a$ is called the \concept{real part} of $z$, and is denoted $\Real(z)$, or $\Re(z)$. The number $b$ is called the \concept{imaginary part} of $z$, and is denoted $\Imaginary(z)$ or $\Im(z)$.
			\end{definition}
			\begin{definition}{The Complex Plane}
				A complex number $z = a + bi$ can be associated with an ordered pair of real numbers $(a, b)$. These numbers can then be represented geometrically by a point in the $xy$-plane. This plane is called the \concept{complex plane}. Points on the $x$-axis are real numbers, as they have an imaginary part of $0$, and points on the $y$-axis are \concept{pure imaginary numbers}, as they have a real part of $0$.
				\begin{center}
					\begin{tikzpicture}
						\begin{axis}[xmin=-1, xmax=1, ymin=-1, ymax=1, axis lines=center, xlabel={Real}, ylabel={Imaginary}, every axis y label/.style={at=(current axis.above origin),anchor=south}, every axis x label/.style={at=(current axis.right of origin),anchor=west}, ticks=none, scale=0.6]
						\end{axis}
					\end{tikzpicture}
				\end{center}
			\end{definition}
			\begin{definition}{Complex Conjugate}
				If $z = a + bi$ is a complex number, then the \concept{complex conjugate} of $z$ is denoted $\bar{z}$, and is defined by:
				\begin{align*}
					\bar{z} = a - bi
				\end{align*}

				You flip the sign of the imaginary part. This results in a vector that is reflected about the real axis.
			\end{definition}
			\begin{definition}{Modulus}
				The producr of a complex number $z = a + bi$ and its conjugate $\bar{z} = a - bi$ is a nonnegative neal number.
				\begin{align*}
					z\bar{z} = (a + bi)(a - bi) = a^{2} - abi + bai - b^{2}i^{2} = a^{2} + b^{2}
				\end{align*}

				The square root of the above value is the length of the vector corresponding to $z$. This is called the \concept{modulus}, or \concept{absolute value} of $z$, and is written $\left\lvert z \right\rvert$
				\begin{align*}
					\left\lvert z\right\rvert = \sqrt{z \bar{z}} = \sqrt{a^{2} + b^{2}}
				\end{align*}
			\end{definition}
			\begin{definition}{Recipricols and Division}
				If $z \neq 0$, then the \concept{recipricol} (or \concept{multiplicative inverse}) of $z$ is denoted by $1/z$ or $z^{-1}$, and is defined such that:
				\begin{align*}
					\frac{1}{z}z = 1
				\end{align*}
				This equation has a unique solution for $1/z$:
				\begin{align*}
					\frac{1}{z} = \frac{\bar{z}}{\left\lvert z\right\rvert^{2}}
				\end{align*}
			\end{definition}
			\begin{definition}{Quotient}
				If $z_{2} \neq 0$, then the \concept{quotient} $z_{1}/z_{2}$ is defined to be the product of $z_{1}$ and $1/z_{2}$
				\begin{align*}
					\frac{z_{1}}{z_{2}} = \frac{\overline{z_{2}}}{\left\lvert z_{2}\right\rvert^{2}}z_{1} = \frac{z_{1}\overline{z_{2}}}{\left\lvert z_{2}\right\rvert^{2}}
				\end{align*}

				The practical way to perform division of complex numbers is therefore to multiply both the top and bottom by the conjugate of the bottom.
			\end{definition}
			\pagebreak
			\begin{sidenote}{Conjugate Rules}
				For any complex numbers $z$, $z_{1}$ and $z_{2}$:
				\begin{enumerate}[label=(\alph*)]
					\item $\overline{z_{1} + z_{2}} = \overline{z_{1}} + \overline{z_{2}}$
					\item $\overline{z_{1} - z_{2}} = \overline{z_{1}} - \overline{z_{2}}$
					\item $\overline{z_{1}z_{2}} = \overline{z_{1}} \cdot \overline{z_{2}}$
					\item $\overline{z_{1}/z_{2}} = \overline{z_{1}} / \overline{z_{2}}$
					\item $\overline{\overline{z}} = z$
					\item $z + \overline{z} = 2 \Real{z}$
					\item $z - \overline{z} = 2 \Imaginary{z}$
					\item $z\overline{z}$ is always real.
				\end{enumerate}
			\end{sidenote}
			\begin{sidenote}{Modulus Rules}
				For any complex numbers $z$, $z_{1}$ and $z_{2}$:
				\begin{enumerate}[label=(\alph*)]
					\item $\left\lvert \overline{z}\right\rvert = \left\lvert z\right\rvert $
					\item $\left\lvert z_{1}z_{2}\right\rvert = \left\lvert z_{1}\right\rvert \left\lvert z_{2}\right\rvert $
					\item $\left\lvert z_{1}/z_{2}\right\rvert = \left\lvert z_{1}\right\rvert /\left\lvert z_{2}\right\rvert $
					\item $\left\lvert z_{1} + z_{2}\right\rvert \leq \left\lvert z_{1}\right\rvert + \left\lvert z_{2}\right\rvert $
					\item $z\overline{z} = \left\lvert z\right\rvert^{2}$
				\end{enumerate}
			\end{sidenote}

		\section{Polar Form}
			\begin{definition}{Polar Form}
				If $z = a + bi$ is a nonzero complex number, and $\phi$ is an angle from the real axis to the vector $z$, then the real and imaginary parts of $z$ can be expressed as:
				\begin{alignat*}{2}
					a &= \left\lvert z\right\rvert \cos \phi & \qquad b &= \left\lvert z\right\rvert \sin \phi
				\end{alignat*}

				The complex number $z = a + bi$ can therefore be expressed as:
				\begin{align*}
					z = \left\lvert z\right\rvert (\cos \phi + i \sin \phi)
				\end{align*}

				The angle $\phi$ is called an \concept{argument} of $z$. Is not unique, but only one argument (the \concept{principal argument}), which will satisfy in radians:
				\begin{align*}
					- \pi < \phi \leq \pi
				\end{align*}
			\end{definition}
			\begin{sidenote}{Geometric Interpretation}
				\begin{itemize}
					\item Multiplying two complex numbers has the geometric effect of multiplying their moduli, and adding their arguments.
					\item Dividing two complex numbers has the geometric effect of dividing their moduli, and subtracting their arguments.
				\end{itemize}
			\end{sidenote}
			\begin{theorem}{De Moivre's Theorem}
				If $n$ is a positive integer, and if $z$ is a nonzero complex number with polar form:
				\begin{align*}
					z = \left\lvert z\right\rvert (\cos \phi + i \sin \phi)
				\end{align*}
				Then:
				\begin{align*}
					z^{n} = \left\lvert z\right\rvert^{n} (\cos (n \phi) + i \sin (n \phi)) 
				\end{align*}
			\end{theorem}
			\begin{theorem}{Binomial Theorem and Pascal's Triangle}
				A method to expand any expression that has been raised to a power. Used to determine the coefficents of a term when expanded.

				\concept{Pascal's Triangle} is a way to determine the terms:
					\begin{center}
						\begin{tikzpicture}[rotate=-90]
							\foreach \x in {0,1,...,5}
							{
								\foreach \y in {0,...,\x}
								{
									\pgfmathsetmacro\binom{factorial(\x)/(factorial(\y)*factorial(\x-\y))}
									\pgfmathsetmacro\shift{\x/2}
										\node[xshift=-\shift cm] at (\x,\y) {\pgfmathprintnumber\binom};
								}
							}
					\end{tikzpicture}
				\end{center}
			\end{theorem}
			\pagebreak
			\begin{example}
				\question{Express $\cos(4\theta)$ and $\sin(4\theta)$ in terms of powers of $\sin\theta$ and $\cos\theta$}
				\begin{enumerate}
					\item Find $(\cos\theta + i\sin\theta)^{4}$ using De Moivre:
						\begin{align*}
							(\cos\theta + i\sin\theta)^{4} = \cos(4\theta) + i\sin(4\theta)
						\end{align*}
					\item Use binomial expansion of $(\cos\theta + i\sin\theta)^{4}$
						\begin{align*}
							(cos\theta + i\sin\theta)^{4} &= \cos^{4}\theta + 4i\cos^{3}\theta\sin\theta - 6cos^{2}\theta\sin^{2}\theta - 4i\cos\theta\sin^{3}\theta + \sin^{4}\theta\\
							&= (cos^{4}\theta - 6\cos^{2}\theta\sin^{2}\theta + \sin^{4}\theta) + i(4\cos^{3}\theta\sin\theta - 4\cos\theta\sin^{3}\theta)
						\end{align*}
					\item Equate the real and nonreal parts.
						\begin{align*}
							\cos(4\theta) &= cos^{4}\theta - 6\cos^{2}\theta\sin^{2}\theta + \sin^{4}\theta\\
							\sin(4\theta) &= 4\cos^{3}\theta\sin\theta - 4\cos\theta\sin^{3}\theta
						\end{align*}
				\end{enumerate}
			\end{example}
			\rulechapterend
\end{document}