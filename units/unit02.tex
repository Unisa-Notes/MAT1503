\providecommand{\main}{..}
\documentclass[\main/notes.tex]{subfiles}

\begin{document}
	\setcounter{chapter}{1}
	\chapter{Determinants}
		\section{Determinants by Cofactor Expansion}
			\begin{definition}{Minors and Cofactors}
				If $A$ is a square matrix, then the \concept{minor of entry $a_{ij}$} is denoted $M_{ij}$, and is defined to be the determinant of the submatrix that remains after the $i$th row and the $j$th column are deleted from $A$.

				The number $(-1)^{i + j}M_{ij}$ is denoted by $C_{ij}$, and is called the \concept{cofactor of entry $a_{ij}$}.
			\end{definition}
			\begin{example}
				Find the minor and cofactor of entry $a_{11}$ in the matrix:
				\begin{align*}
					A = \begin{bmatrix}
						3 & 1 & -4\\
						2 & 5 & 6\\
						1 & 4 & 8
					\end{bmatrix}
				\end{align*}

				To find the minor, remove the row and column indicated, so the first row and column. This gives us the matrix:
				\begin{align*}
					M_{11} = \begin{vmatrix}
						5 & 6\\
						4 & 8
					\end{vmatrix} = 16
				\end{align*}

				To find the cofactor:
				\begin{align*}
					C_{11} &= (-1)^{1 + 1}M_{11}\\
					&= M_{11}\\
					&= 16
				\end{align*}
			\end{example}
			\begin{sidenote}{Cofactors use a checkerboard of plus and minus}
				\begin{align*}
					\begin{bmatrix}
						+ & - & + & - & + & \cdots\\
						- & + & - & + & - & \cdots\\
						+ & - & + & - & + & \cdots\\
						- & + & - & + & - & \cdots\\
						\vdots & \vdots & \vdots & \vdots & \vdots
					\end{bmatrix}
				\end{align*}
			\end{sidenote}
			\begin{definition}{General Determinant}
				If $A$ is an $n \times n$ matrix, then the number obtained by multiplying the entries in any row or column of $A$ by the corresponding cofactors, and adding the resulting products is called the \concept{determinant of $A$}, and the sums themselves are called the \concept{cofactor expansions of A}.
				\begin{align*}
					\det(A) &= a_{1j}C_{1j} + a_{2j}C_{2j} + \cdots + a_{nj}C_{nj}\\
					\det(A) &= a_{i1}C_{i1} + a_{i2}C_{i2} + \cdots + a_{in}C_{in}
				\end{align*}
			\end{definition}
			\begin{sidenote}{Determinant of a Triangular Matrix}
				For a triangular matrix, the determinant is just the product of all the entries in the main diagonal.
			\end{sidenote}

		\section{Evaluating Determinants by Row Reduction}
			\begin{theorem}{Zero Rows}
				Let $A$ be a square matrix. If $A$ has a row or column of zeros, then $\det(A) = 0$.
			\end{theorem}
			\begin{theorem}{Determinant of the Transpose}
				Let $A$ be a square matrix. Then $\det(A) = \det(A^{T})$.
			\end{theorem}
			\begin{sidenote}{Elementary Row Operations and Determinants}
				\begin{center}
					\begin{tblr}{>{\raggedright}Xl| ll}
						Matrix Operation & Notation & Effect on Determinant & Notation\\
						\midrule
						Multiply a row by $k$ & $kR_{i} \rightarrow R_{i}$ & Multiply determinant by $k$ & $\det(B) = k \cdot \det(A)$\\
						Swap two rows & $R_{i} \leftrightarrow R_{j}$ & Multiply determinant by $-1$ & $\det(B) = - \det(A)$\\
						Add a multiple of a row to another row & $R_{i} + kR_{j} \rightarrow R_{i}$ & Stays the same & $\det(B) = \det(A)$
					\end{tblr}
				\end{center}
			\end{sidenote}
			\pagebreak
			\begin{sidenote}{Elementary Matrices and Determinants}
				Let $E$ be an $n \times n$ elementary matrix.
				\begin{enumerate}[label=(\alph*)]
					\item If $E$ results from multiplying a row of $I_{n}$ by a nonzero number $k$, then $\det(E) = k$
					\item If $E$ results from interchanging two rows of $I_{n}$, then $\det(E) = -1$
					\item If $E$ results from adding a multiple of one row of $I_{n}$ to another, then $\det(E) = 1$
				\end{enumerate}
			\end{sidenote}
			\begin{theorem}{Proportional Rows or Columns}
				If $A$ is a square matrix with two or more proportional rows or columns, then $\det(A) = 0$.
			\end{theorem}
			\begin{definition}{Evaluating Determinants Using Row Reduction}
				\begin{enumerate}
					\item Simplify the matrix into triangular form. Remember the operations done to transform it.
					\item Find the determinant of the triangular matrix, by multiplying the diagonals.
					\item Multiply that determinant by the different factors used to manipulate it.
				\end{enumerate}
			\end{definition}

		\section{Properties of Determinants}
			\begin{theorem}{Determinants and Multiplying a Matrix by a Scalar}
				Given a square matrix $A$ of size $n \times n$, and a scalar $k$, then
				\begin{align*}
					\det(kA) = k^{n}\det(A)
				\end{align*}
			\end{theorem}
			\begin{sidenote}{Sums of Determinants}
				It is \emph{not} always the case for determinants that $\det(A + B) = \det(A) + \det(B)$
			\end{sidenote}
			\begin{theorem}{Sums of Determinant Exception}
				Let $A$, $B$ and $C$ be $n \times n$ matrices that differ only in a single row (eg. the $r$th row). Assume that the row (the $r$th row) of $C$ can be obtained by adding corresponding entries in the $r$th rows of $A$ and $B$. Then
				\begin{align*}
					\det(C) = \det(A) + \det(B)
				\end{align*} 
				The same result holds for columns.
			\end{theorem}
			\begin{theorem}{Determinant of a Matrix Product}
				If $A$ and $B$ are square matrices of the same size, then
				\begin{align*}
					\det(AB) = \det(A) \cdot \det(B)
				\end{align*}
			\end{theorem}
			\begin{theorem}{Determinant of an Invertible Matrix}
				If $A$ is invertible, then
				\begin{align*}
					\det(A^{-1}) = \frac{1}{\det(A)}
				\end{align*}
			\end{theorem}
			\begin{definition}{Matrix of Cofactors and Adjoints}
				If $A$ is any $n \times n$ matrix, and $C_{ij}$ is the cofactor of $a_{ij}$, then the matrix:
				\begin{align*}
					\begin{bmatrix}
						C_{11} & C_{12} & \cdots & C_{1n}\\
						C_{21} & C_{22} & \cdots & C_{2n}\\
						\vdots & \vdots & \ddots & \vdots\\
						C_{n1} & C_{n2} & \cdots & C_{nn}
					\end{bmatrix}
				\end{align*}
				is called the \concept{matrix of cofactors from A}. The transpose of this matrix is called the \concept{adjoint}, and is written $\adj(A)$.
			\end{definition}
			\begin{theorem}{Inverse of a matrix using its adjoint}
				If $A$ is an invertible matrix, then
				\begin{align*}
					A^{-1} = \frac{1}{\det(A)} \cdot \adj(A)
				\end{align*}
			\end{theorem}

		\section{Cramer's Rule}
			\begin{definition}{Cramer's Rule}
				If $A\mathbf{x} = \mathbf{b}$ is a system of $n$ linear equations in $n$ unknowns such that $\det(A) \neq 0$, then the system has a unique solution. The solution is:
				\begin{align*}
					x_{1} &= \frac{\det(A_{1})}{\det(A)}\\
					x_{2} &= \frac{\det(A_{2})}{\det(A)}\\
					&\vdots\\
					x_{n} &= \frac{\det(A_{n})}{\det(A)}
				\end{align*}
				where $A_{j}$ is the matrix obtained by replacing the entries in the $j$th column of $A$ by the entries in the solution matrix $\mathbf{b}$:
				\begin{align*}
					\mathbf{b} = \begin{bmatrix}
						b_{1}\\
						b_{2}\\
						\vdots\\
						b_{n}
					\end{bmatrix}
				\end{align*}
			\end{definition}

		\section{Eigenvectors and Eigenvalues}
			% TODO: COME BACK TO

	\rulechapterend
\end{document}